% Paquetes de generalidades
%%%%%%%%%%%%%%%%%%%%%%%%%%%%%%%

%-----------------------------------------------------------
% defino el formato del documento
\documentclass[a4paper,12pt]{article}
\usepackage[utf8]{inputenc}
\usepackage[spanish]{babel}
\usepackage{csquotes}
\usepackage[a4paper]{geometry}%
\geometry{
  a4paper,
  left       =10mm,
  right      =10mm,
  top        =30mm,
  bottom     =15mm,
  headheight =20mm
}
%-----------------------------------------------------------

% Paquetes para matemática
%%%%%%%%%%%%%%%%%%%%%%%%%%%%%%%

% Los paquetes ams son desarrollados por la American Mathematical Society y mejoran la escritura de fórmulas y símbolos matemáticos.
\usepackage{amsmath}
\usepackage{amsfonts}
\usepackage{amssymb}

% Paquetes para manejo de gráficas y figuras
%%%%%%%%%%%%%%%%%%%%%%%%%%%%%%%


% para tabular graficos tablas y ecuaciones
\usepackage{tabularx}

\usepackage{multirow}


% Para insertar gráficas
\usepackage{graphicx}

% Para colocar varias subfiguras
\usepackage[lofdepth,lotdepth]{subfig}

% Para crear gráficos vectoriales con un lenguaje descriptivo/geométrico
\usepackage{tikz}

% Para crear circuitos vectoriales basados en TikZ
\usepackage[american]{circuitikz}

% Paquetes relacionados con el estilo
%%%%%%%%%%%%%%%%%%%%%%%%%%%%%%%

% Para la presentación correcta de magnitudes y unidades
\usepackage{siunitx}

% Para hipervínculos y marcadores
\usepackage[colorlinks=true,urlcolor=blue,linkcolor=black,citecolor=green]{hyperref}
	\urlstyle{same}

% Para ubicar las tablas y figuras justo después del texto
\usepackage{float}

% Para hacer tablas más estilizadas
\usepackage{booktabs}

% Para hacer secciones con múltiples columnas
\usepackage{multicol}

% Para insertar código fuente estilizado
\usepackage{listings}
	\lstset{basicstyle=\ttfamily,breaklines=true}
    \lstset{numbers=left, numberstyle=\tiny, stepnumber=1, numbersep=6pt}

% Para agregar código con formato de Matlab
\usepackage[numbered,autolinebreaks]{mcode}

% Para utilizar el número de páginas
\usepackage{lastpage}

% Para manejar los encabezados y pies de página
\usepackage{fancyhdr}
	% Contenido de los encabezados y pies de pagina
	\pagestyle{fancy}

%Para cambiar la orientación del texto
\usepackage[document]{ragged2e}

% Misceláneos
%%%%%%%%%%%%%%%%%%%%%%%%%%%%%%%
\usepackage{enumitem}
\usepackage{verbatim}


% Para insertar símbolos extraños
\usepackage{marvosym}

\usepackage{pdfpages}

