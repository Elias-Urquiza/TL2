

\documentclass[a4paper, 10pt, spanish]{article}
\usepackage{color}
\definecolor{cadet}{rgb}{0.33, 0.41, 0.47}
\definecolor{orange}{rgb}{0.93, 0.53, 0.18}
\definecolor{carminered}{rgb}{1.0, 0.0, 0.22}
\definecolor{green}{rgb}{0.33, 0.42, 0.18}
\definecolor{darkmagenta}{rgb}{0.55, 0.0, 0.55}
\usepackage{anysize}
\usepackage{amsmath}
\usepackage{biblatex}
\usepackage{float}
\usepackage{array} % 1
\usepackage{graphicx}
\usepackage{tikz}
\usetikzlibrary{calc,patterns,angles,quotes}
\usetikzlibrary{calc,decorations.pathmorphing,patterns}
\usepackage{graphicx}
\usepackage[spanish]{babel}
\usepackage[T1]{fontenc}
\usepackage[utf8]{inputenc}
\usepackage{textcomp}
\usepackage{fancyhdr}
\usepackage{color}
\usepackage{courier}
\usepackage{multirow}
\usepackage{float}
\usepackage{listings}
\usepackage{pgfplots,filecontents}
\pgfplotsset{compat=1.7}
\pgfplotsset{compat=newest}
\usepgfplotslibrary{units}
\usepackage[siunitx]{circuitikz}
\usepackage{caption}
\usepackage{subcaption}
\usepackage{cleveref}
\usepackage{tabularx}
\usepackage{lscape}
\usepackage{pdflscape}
\usepackage{booktabs}


\usepackage{lastpage}   % Para poder saber cuántas páginas tiene el documento
\pagestyle{fancy}
\renewcommand{\sectionmark}[1]{\markboth{}{\thesection\ \ #1}}
\fancyhead{}	% Elimino el contenido del encabezado
% El siguiente texto a la derecha (izquierda) en páginas pares (impares)
\fancyhead[L]{86.06 - Circuitos Electrónicos - Informe N\textsuperscript{o}4}
\fancyhead[R]{FIUBA}

\fancyfoot{}	% Elimino el contenido del pie de página
% A la izquierda (derecha) en páginas pares (impares): nro. de página / total
\fancyfoot{\thepage/\pageref{LastPage}}



\begin{document}


\marginsize{2cm}{2cm}{2cm}{2cm}
%
% Carátula:
%
\begin{titlepage}

\thispagestyle{empty}

\begin{center}
\includegraphics[scale=0.3]{fiuba.pdf}\\
\large{\textsc{Universidad de Buenos Aires}}\\
\large{\textsc{Facultad de Ingeniería}}\\
% Modificar año y cuatrimestre
\small{Año 2019 - 2\textsuperscript{o} cuatrimestre}
\end{center}

\vfill

\begin{center} % Modificar el código de ser necesario
\Large{\underline{\textsc{Circuitos Electrónicos (86.06)}}}\\ \vspace{0.5cm}
\Large{\underline{\textsc{Diseño de Circuito Amplificador Para Sensor de Laboratorio}}}\\ \vspace{0.5cm}
\Large{\underline{\textsc{Informe de Laboratorio N\textsuperscript{o}~4}}}
\end{center}

\vfill

\begin{center}
\large{José F. González - 100063 - \footnotesize{\verb!<jfgonzalez@fi.uba.ar>!}}\\ \vspace{0.25cm}
\large{Gottfried, Joel - 102498 - \footnotesize{\verb!<joelgottfried99@gmail.com>!}}\\\vspace{0.25cm}
\large{Urquiza, Elias - 100714 - \footnotesize{\verb!<eurquiza@fi.uba.ar>!}}\\
\end{center}

\vfill

\hrule
\vspace{0.2cm}

% Modificar código de ser necesario
\noindent\small{86.06 - Circuitos Electrónicos \hfill }

\end{titlepage}

%
% Pongo el índice en una página aparte:
%
\tableofcontents
\newpage


\section{Objetivos}
    Se busca crear un circuito capaz de adaptar la diferencia de potencial producida por un sensor de fuerza FN3030 al rango de tensiones que maneja el ADC de un arduino (0-5V). El mismo debe contar con una etapa diferencial en la entrada ya que la señal producida por el sensor posee ruido de amplitud comparable a la de la señal.
    También se busca que la ganancia sea ajustable ya que no se conoce con precisión la fuerza máxima que va a recibir el sensor medible, y se busca maximizar la eficiencia del ADC.

\section{Alimentación}

    Dado que se busca que el circuito opere en un ambiente pequeño se decidió crear una fuente de alimentación especifica para el circuito. %La misma es una fuente partida a partir de un transformador 220v-30v cuyo circuito puede verse en la figura XX.

\section{Diseño}


    Dado que el sensor de fuerza incluye un amplificador diferencial AD620, se utilizó el mismo para la etapa diferencial. La ganancia del mismo es configurable mediante la ecuación~\ref{eq:ganancia}. Se utilizó un amplificador operacional TL082 en configuración no inversora en conjunto con un preset para crear la ganancia ajustable mencionada previamente.

    Se diseñó el circuito de la figura \ref{fig:circuito}.

    \begin{equation}\label{eq:ganancia}
        G = \frac{49.4K\Omega}{R_g} + 1
    \end{equation}

    \begin{center}
      \includegraphics[width=0.8\textwidth]{Circuito.png}
      \captionof{figure}{Circuito diseñado.}
      \label{fig:circuito}
    \end{center}

\newpage

\section{Mediciones}

    Se realizaron una serie de mediciones utilizando pesas de alta precisión y se buscó aproximar los datos obtenidos mediante una recta regresión. Esta recta corresponde a la tensión en función de la fuerza, que en este caso está constituida puramente por la fuerza peso de las pesas. Lo primero que se hizo fue medir la máxima fuerza que puede recibir el sensor y ajustar la ganancia del amplificador tal que la tensión de salida máxima del circuito no supere la tensión límite de entrada (5V) del arduino.

    Las mediciones pueden verse en la tabla~\ref{tb:mediciones}

    \begin{center}
        \begin{tabular}{| l | l |}
        \hline
        Peso & Tensión medida \\ \hline
        0g   & 241mV  \\ \hline
        20g  & 318mV  \\ \hline
        50g  & 433mV  \\ \hline
        100g & 625mV  \\ \hline
        200g & 1009mV \\ \hline
        250g & 1.2V   \\ \hline
        500g & 2.16V  \\ \hline
        \end{tabular}
        \captionof{table}{Tabla de resultados\label{tb:mediciones}}
    \end{center}

     Mediante el uso de cuadrados mínimos se obtuvo la recta deseada, que puede verse en la ecuación~\ref{eq:recta}, donde F es la fuerza que esta siendo aplicada sobre el sensor y V es la tensión medida en milivolts.

    \begin{equation}\label{eq:recta}
        F = V\cdot0.259 - 60.456
    \end{equation}


\section{Conclusiones}
Podemos observar que el circuito diseñado cumple con las características deseadas para realizar mediciones con el sensor de peso. Medimos, además, un error cuadrático medio de 5.5mV entre la curva de regresión por mínimos cuadrados y los datos medidos.  

\end{document}
